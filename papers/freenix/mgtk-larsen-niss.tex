\documentclass[workingdraft,endnotes]{usetex-v1}
% 1. workingdraft:
%
%       For initial submission and shepherding.  Features prominent
%       date, notice of draft status, page numbers, and annotation
%       facilities.  The three supported annotation macros are:
%               \edannote{text}         -- anonymous annotation note
%               \begin{ednote}{who}     -- annotation note attributed
%                 text                          to ``who''
%               \end{ednote}
%               \HERE                   -- a marker that can be left
%                                               in the text and easily
%                                               searched for later
%
% In addition, the option "endnotes" permits the use of the
% otherwise-disabled, Usenix-deprecated footnote{} command in
% documents.  In this case, be sure to include a
% \makeendnotes command at the end of your document or
% the endnotes will not actually appear.
%
\usepackage{url}
\usepackage{graphicx}





\begin{document}

\title{mGTK}

\docstatus{Submitted to USENIX'04 (FREENIX track)}

\author{
\authname{Ken Friis Larsen}
%\authaddr{Your Department}
%\authaddr{Your Institution}
%\authaddr{ Your City, State, ZIP}
\authurl{\url{ken@friislarsen.net}}
%\authurl{\url{http://host.dom/yoururl}}
\and
\authname{Henning Niss}
\authaddr{Department of Innovation}
\authaddr{IT University of Copenhagen}
\authaddr{Denmark}
\authurl{\url{hniss@it.edu}}
\authurl{\url{http://www.it.edu/people/hniss}}
%
} % end author

\maketitle

\begin{abstract}
  mGTK is glue code to make GTK+ accessible from Standard ML. This
  provides a convenient way for SML programmers to add GUI features to
  applications.
\end{abstract}


\section{Why Standard ML and GTK+}

\begin{ednote}{KFL}
  \begin{itemize}
  \item Hvad og hvorfor: SML  
  \item Gtk+ er interessant (og C var et godt
    impl. valg)  
  \item SML er ikke OO (tester Gtk+-folkenes "hypotese")
\end{itemize}
\end{ednote}

Standard ML (SML) is a functional language with imperative features
widely used for teaching and in reasearch. 



\section{Related Work}
\label{sec:related-work}

 Brainstorm of related work:
\begin{itemize}
\item SML-GTK 
\url{http://www.cs.nyu.edu/phd_students/leunga/sml-gtk/sml-gtk.html}

\item gtk+hs
\url{http://www.cse.unsw.edu.au/~chak/haskell/gtk}

\item lablgtk
\url{http://wwwfun.kurims.kyoto-u.ac.jp/soft/olabl/lablgtk.html}

\item sml\_tk
\url{http://www.informatik.uni-bremen.de/~cxl/sml_tk}

\item erlgtk
\url{http://erlgtk.sourceforge.net}

\end{itemize}



\section{Future Work}
\label{sec:future-work}

\begin{itemize}
\item Wrap all the libraries in the GNOME development platform
\end{itemize}



\section{Conclusion}
\label{sec:conclusion}

In this paper we have demonstrated that is theoretical and practically
possible to make an interface from SML to GTK.


\end{document}

