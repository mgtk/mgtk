\documentclass{sig-alternate}

\usepackage{preamble}

\begin{document}

% Title page
\title{mGTK}% you can use \titlenote inside \title
%\subtitle
\numberofauthors{2}
\author{
  \alignauthor Ken Friis Larsen\\
     \affaddr{The IT University of Copenhagen}\\
     \affaddr{DENMARK}\\
     \email{kfl@it-c.dk}
  \alignauthor Henning Niss\\
     \affaddr{The IT University of Copenhagen}\\
     \affaddr{DENMARK}\\
     \email{hniss@it-c.dk}
}
\date{March 2003}
\maketitle

\begin{abstract}
foo bar
\end{abstract}

\category{H.4}{Information Systems Applications}{Miscellaneous}
\category{D.2.8}{Software Engineering}{Metrics}[complexity measures, performance measures]

%\terms

\keywords{Standard ML, language binding, \Gtk, Gnome platform}

\section{Introduction}

A key problem raised against functional programming languages again and
again is the lack of library support necessary for real-world application
developement. Example libraries include toolkits for building graphical
user interfaces, infrastructure that allows the application to be
decomposed into components and use standard components for common tasks,
database connectivity, XML parsers, processors and so on, and the
list continues.

We are presently in the process of providing some of these libraries
in the context of Standard ML, more specifically
\mosml~\cite{Mosml-webpage:2003} and the ML Kit (in
progress)~\cite{MLKit-webpage:2003} and hopefully also
MLton~\cite{MLton-webpage:2003}. The long term goal of our project is
to provide access and interoperability with the \GNOME
platform~\cite{GNOME-webpage:2003}. In this paper we report on parts
of that goal: a binding for the graphical toolkit
\Gtk~\cite{Gtk-webpage:2003} for the above mentioned SML systems.

\writeme{the ``platform/infrastructure nature'' of the project}

\section{The \glib and \Gtk libraries}

\writeme{brief overview of \glib and \Gtk}

\subsection{GObjects}

\subsection{GValues}

\subsection{Closures}

\subsection{Widget hierarchy}

\subsection{Signals}

\section{Object hierarchies and phantom types}
\label{sec:representing-object-hierarchies}

\writeme{show how to represent single inheritance object hierarchies
  using phantom types}

\writeme{hmm, here is a reference that might prove interesting
  \cite{Fluet-Pucella:2002}.}

\section{Interfacing memory management}

\section{GValues and type dynamic}
\label{sec:representing-gvalues}

\writeme{give Ken's intuition about GValues and type dynamic}

\section{Callbacks}
\label{sec:representing-callbacks}

\section{Stubs and code generation}

The bulk part of \mGTK is merely stub code, on both the C- and
SML-sides, converting between different representations of objects.
How to write such code is not terribly interesting and it is discussed
elsewhere \cite{Larsen:2001}. Luckily, most of \mGTK is machine
generated. In fact, once the issues of representing the object
hierarchy, representing GValues, and calling back and forth between
SML and C
(Sections~\ref{sec:representing-object-hierarchies}--\ref{sec:representing-callbacks})
have been settled, the rest of the work is automatic.

\Gtk has, right from the beginning, been addressing the possibility of
\emph{binding} the toolkit for other languages than C. Concretely, a
``specification'' of the toolkit is readily available and from this it
is relatively straightforward to generate a binding for another
language. The specification specifies objects by listing their place
in the object hierarchy and their fields with associated types;
methods, functions, and signals by listing their argument types and
result types; enumerations and flags by listing their members; and
so on.

The specifications are found in so-called ``defs files'' (the format
being defined in \cite{defs-format:2003}). Here is an example
specification of a method
\begin{Verbatim}
(define-function gtk_button_new_with_label
  (c-name "gtk_button_new_with_label")
  (return-type "GtkWidget*")
  (parameters
    '("const-gchar*" "label")
  )
)
\end{Verbatim}

In practice, the code generation is based on a small muck-up of the
toolkit with a few widgets and associated methods. The muck-up is
intended to allow us to experiment with the concrete representing
without burdening the experiments with numerous (insignificant)
details. It also allow us to easily the extend the binding to other
SML systems if they can provide us with the muck-up. Based on the 
muck-up the code generation simply mimics the structure for the
complete toolkits.

\writeme{include some numbers here}


\section{Conclusion}

\writeme{future work}

\writeme{related work}

\bibliographystyle{abbrv}
\bibliography{mgtk}

\end{document}