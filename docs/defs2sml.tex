\documentclass{article}

\usepackage{preamble}

\title{{\large {\libname} --- an SML binding for Gtk}\\
       Generator Documentation}
\author{Henning Niss\and Ken Friis Larsen}
\date{July 2000}

\begin{document}

\maketitle

\begin{abstract}
This document contains some brief comments about the workings of the
program that generates the {\libname} binding files
(\filename{Gtk.sig}, \filename{Gtk.sml}, and \filename{mgtk.c}). The
program is called {\progname} and we shall refer to it as ``the
generator'' for lack of a better name.

\begin{center}
\textbf{This information is terribly outdated!}
\end{center}
\end{abstract}

\tableofcontents

\section{Phases of the generation}

In this section we explain how the generation of the stub files for a
library like {\libname} proceeds. This is useful for the explanation
of the different files constituting the generator.

The generator program is run three times: to produce the C stub code,
the signature file(s), and the structure file(s). Each of these runs
proceed in the same fashion, except of course that the resulting
output differs.

The input to {\progname} is most importantly an
\emph{interface definition file}. In the {\gtkname} world such a file
has extension \filename{.defs} and is composed of a series of
\emph{declarations} of types and functions.  The file is written in a
Scheme-like syntax and is consequently very easy to parse.

The two most important phases in a run of {\progname} are
\begin{enumerate}
\item The \filename{.defs} file is lexed and parsed to give an
  \emph{abstract syntax}. All names have been split into their
  constituent parts---for example, \function{gtk_button_new} is split
  into the \emph{module path} (\syntax{gtk},\syntax{button}) and the
  \emph{base} \syntax{new}. Similarly, all type expressions have been
  parsed into a tree-like representations.
  
\item From the abstract syntax \emph{target code} is generated. Target
  code is currently represented by efficiently concatenable strings
  (called \emph{wseq}'s).  [It may make more sense to generate a more
  abstract form of target code to later be pretty printed.]
\end{enumerate}

\section{Roadmap (file overview)}

Based on the separation of {\progname}'s functionality into phases it
is easy to divide the files constituting {\progname} into three
groups: (a) front end (lexer/parser) files, (c) code generation files,
and (c) utility files. The utility files contain functions for
efficiently concatenable strings, managing names, parsing command-line
arguments, and similar things.

\begin{description}
\item[]\mbox{}\vspace{-.5cm}%

  \begin{filedescription}
     {\normalfont\bfseries\itshape File name} & \textbf{\textit{Purpose}} \\
     \hline
  \end{filedescription}

\item[(a) front end files]
  %
  The parser is based on \emph{parser combinators}. In fact everything
  related to parsing is constructed from the combinators---even the
  lexer. The parser uses very simple combinations of combinators
  (witness the definitions of \function{mdlDecl}, \function{objDecl},
  \function{fncDecl}, \function{enumDecl}, \function{flagsDecl}, and
  \function{signalDecl} in \filename{Parser.sml}).  Everything is
  parsed into abstract syntax---lists of declarations.  During parsing
  names are split into a module path and a base and type expressions
  are constructed.

  \begin{filedescription}
     Parsercomb.\{sig,sml\}
             & Parser combinators (from the MosML 2.0 distribution) \\
     Parser.sml
             & The parser specification \\
     ParseUtils.\{sig,sml\}
             & Utility functions for constructing abstract syntax \\
     Parse.sml
             & A single function stringing together all parser issues \\
     AST.\{sig,sml\}
             & Abstract syntax and operations on it \\
     TypeExp.\{sig,sml\}
             & Type expressions and operations on them
  \end{filedescription}

\item[(b) code generation files]
   \mbox{}%

   \begin{filedescription}
      Translate.\{sig,sml\}
              & The main code generator \\
      TypeInfo.sml
              & Information about types (how to convert to/from C, 
                how to construct C/SML type names, etc) \\
      Messages.\{sig,sml\}
              & Functions for generating messages in the target code 
   \end{filedescription}

\item[(c) utility files]
   \mbox{}%

   \begin{filedescription}
      ArgParse.\{sig,sml\}
              & Parsing of command-line arguments \\
      WSeq.\{sig,sml\}
              & Efficiently concatenable strings (from the MosML 2.0 
                distribution) \\
      NameUtil.\{sig,sml\}
              & Utilities for handling names (splitting names, 
                upper/lower-casing, etc) \\
      Util.sml
              & Miscellaneous utilities (mostly exception handling) \\
      State.sml
              & Representation of command-line options \\
      Main.sml
              & The main driver binding everything together
   \end{filedescription}

\end{description}

In addition to the main line of files, there are two subdirectories:
\begin{description}
\item[(d) regression ]
  
  This directory contains a small regression test for the {\progname}
  program. It works by running {\progname} on an input file
  three times to produce \filename{.c}, \filename{.sig}, and
  \filename{.sml} files which are then concatenated to one big file.
  This resulting file (called \filename{.res}) is the
  \command{diff}'ed against a trusted version (called
  \filename{.trusted}). The \filename{.res} files can safely be
  deleted once checked against \filename{.trusted}.

   \begin{filedescription}
     $*$.defs     & Input test examples \\ 
     $*$.trusted  & Verified output of the test examples \\ 
     Makefile   & Driver Makefile
   \end{filedescription}

\item[(e) tools]
  
  This directory contains various tools inspecting .defs files.

   \begin{filedescription}
     defsdiff.sml  & Compare two \filename{.defs} files; \emph{not finished} \\
     defsfind.sml  & Find a declaration in a \filename{.defs} file \\
     defslist.sml  & List contents of a \filename{.defs} file \\
     defsstat.sml  & A small amount of statistics for a \filename{.defs}
   \end{filedescription}

\end{description}

\section{Invocation}

The command-line options recognized by the generator is described in
this section.

\paragraph{Synopsis:}
\begin{flushleft}
\begin{tabular}{@{}ll@{}}
  {\progname} &
     \option{-c}
     \option{-sig}
     \option{-sml}
     \option[file]{-o}
  \\&
     \option[file]{-h}
     \option[file]{--header}
     \option{--no-header}
  \\&
     \option[file]{-f}
     \option[file]{-footer}
     \option{-end}
     \option{--end-footer}
     \option{--no-footer}
  \\&
     \option{-V}
     \option{--version}
  \\&
     \option{-v}
     \option{--verbose}
  \\&
     \optionarg{file}
\end{tabular}
\end{flushleft}

\paragraph{Description:} The options have the following meanings:
\begin{description}
  
\item \option{-c} Generate C code. Will set header to
  \filename{header.c} (see \option{--no-header} below). This is the
  default target language.
  
\item \option{-sig} Generate an SML signature. Will set header to
  \filename{header.sig} and implies \option{--end-footer} (see
  \option{--no-header} below).
  
\item \option{-sml} Generate an SML structure. Will set header to
  \filename{header.sml} and implies \option{--end-footer} (see
  \option{--no-header} below).
  
\item \option[file]{-o} Sets the output file to
  \optionarg{file}.  All output goes to \optionarg{file}.  The default
  is to output on \filename{stdout}.
  
\item \option[file]{-h} Sets the header file to
  \optionarg{file}. The header file is included after the copyright
  message, but before the generated code. This is implied by
  \option{-c}, \option{-sig}, and \option{-sml} above.

\item \option[file]{--header} Same as \option[file]{-h}.
  
\item \option{--no-header} Don't insert a header. Since \option{-c},
  \option{-sig}, and \option{-sml} automatically sets the header, in
  order to avoid headers with these options, \option{--no-header} must
  appear \emph{after} the target option.
  
\item \option[file]{-f} Sets the footer file to
  \optionarg{file}. The footer file is included after the generated
  code.

\item \option[file]{--footer} Same as \option[file]{-f}.
  
\item \option{-end} Inserts \literal{end} as the footer.  This is
  implied by \option{-sig} and \option{-sml} above to end the
  signature/structure declaration.

\item \option{--end-footer} Same as \option{-end}.
  
\item \option{--no-footer} Don't insert a footer. Since \option{-sig}
  and \option{-sml} automatically sets the footer to \option{-end},
  this option must appear after the target option.

\item \option{-V} Show version information.

\item \option{--version} Same as \option{-V}.
  
\item \option{-v} Display extra (debug) messages. This is not very
  usefull currently since it only tells you which options have been
  seen.

\item \option{--verbose}  Same as \option{-v}.
  
\item \optionarg{file} The \filename{.defs} file to be
  processed.
\end{description}

\section{Accepted syntax}

The generator accepts \filename{.defs} files complying with the
following grammar.

\paragraph{Declarations:}\mbox{}%

\begin{ebnf}
  <decls> &::=& <decl>+                                                    
                                                                           
  <decl> &::=&  <obj-decl>   |  <func-decl>                                
         &|&    <enum-decl>  |  <boxed-decl>  |  <signal-decl>             
\end{ebnf}

\paragraph{Widgets:}\mbox{}%

\begin{ebnf}
  <obj-decl> &::=&  "(" "define-object" <name> <inherits> [<fields>] ")"   
  <inherits> &::=&  "(" <name> ")"                                         
  <fields>   &::=&  "(" "fields" <pars> ")"                                
\end{ebnf}

\paragraph{Functions:}\mbox{}%

\begin{ebnf}
  <func-decl> &::=&  "(" "define-func" <name> <type> <pardefaults> ")"     
\end{ebnf}

\paragraph{Enums/flags:}\mbox{}%

\begin{ebnf}
  <enum-decl> &::=&  "(" "define-enum" <name> <constructors> ")"           
              & | &  "(" "define-flag" <name> <constructors> ")"           
  <constructors> &::=&  <constructor>+                                     
  <constructor>  &::=&  "(" <name> <name> ")"                              
\end{ebnf}

\paragraph{Boxed:}\mbox{}%

\begin{ebnf}
  <boxed-decl> &::=& "(" "define-boxed" <name> <func> <func> [<size>] ")"  
  <func>       &::=&  <name>                                               
  <size>       &::=&  <string>                                             
\end{ebnf}

\paragraph{Signals:}\mbox{}%

\begin{ebnf}
  <signal-decl> &::=& "(" "define-signal" <name> <signame> [<sigtype>] ")" 
  <signame>     &::=&  <string>                                            
  <sigtype>     &::=&  "(" <type> <pars> ")"                               
\end{ebnf}

\paragraph{General stuff:}\mbox{}%

\begin{ebnf}
  <pardefaults> &::=&  "(" <pardefault>* ")"                               
  <pardefault>  &::=&  "(" <type> <name> [<flag>] [<default>] ")"          
  <pars>        &::=&  "(" <par>* ")"                                      
  <par>         &::=&  "(" <type> <name> ")"                               
                                                                           
  <type>    &::=&  <word>                                                  
  <flag>    &::=&  "(" "null-ok" ")"  |  "(" "output" ")"                  
  <default> &::=&  "(" "=" <string> ")"                                    
                                                                           
  <name>    &::=&  a sequence of letters, \syntax{-}, and \syntax{_} beginning with a letter                      
  <word>    &::=&  ditto                                                   
  <string>  &::=&  anything encapsulated in double quotes                  
\end{ebnf}

\medskip\noindent Here\\
\begin{ebnf}
  &$\bullet$&  <nt>*   denotes zero or more occurrences of <nt>  ; 
  &$\bullet$&  <nt>+   denotes one or more occurrences of <nt>  ; 
  &$\bullet$&  [<nt>]  denotes zero or one occurrence of <nt>  .
\end{ebnf}

\medskip

The specification of signals is not standard; all other grammar rules
are induced from the \filename{gtk.defs} file in the
\programname{PyGtk} distribution. However, \programname{PyGtk} does
not specify signals, and consequently a declaration in the style of
the remaining declarations was added.

TODO: 
\begin{itemize}
\item could we improve things by not differentiating between
  \textit{pardefaults} and \textit{pars}?
\end{itemize}


\section{Translation}

This section lists example translations of parts of the current
\filename{gtk.defs} file.

\subsection{Widgets}

The \filename{.defs} file contains
\begin{verbatim}
    (define-object GtkDialog (GtkWindow)
       (fields
          (GtkVBox vbox)
          (GtkHBox action_area)))
\end{verbatim}

To facilitate access to the subwidgets of a widget (which in C is
handled by simply casting the widget pointer to the correct type and
then accessing the fields), we need to provide SML functions that
extract these subwidgets of a widget, since the SML program cannot
simply access the fields.

The corresponding generated code is
\begin{itemize}
\item \filename{.c}
\begin{verbatim}
    /* *** Dialog stuff *** */

    /* ML type: gtkobj -> gtkobj */
    value mgtk_dialog_get_vbox(value wid) { /* ML */
      return Val_GtkObj((GTK_DIALOG(GtkObj_val(wid))) -> vbox);
    }
\end{verbatim}

\item \filename{.sig}
\begin{verbatim}
    (* *** Dialog *** *)

    type 'a dialog_t
    type 'a GtkDialog = 'a dialog_t GtkWindow

    val dialog_get_vbox: 'a GtkDialog -> base GtkVBox
    ...
\end{verbatim}

\item \filename{.sml}
\begin{verbatim}
    (* *** Dialog *** *)

    type 'a dialog_t = base
    type 'a GtkDialog = 'a dialog_t GtkWindow

    val dialog_get_vbox_: gtkobj -> gtkobj
        = app1(symb"mgtk_dialog_get_vbox")
    val dialog_get_vbox: 'a GtkDialog -> base GtkVBox
        = fn OBJ wid => OBJ(dialog_get_vbox_ wid)

    ...
\end{verbatim}

\end{itemize}


\subsection{Functions}

The \filename{.defs} file contains
\begin{verbatim}
    (define-func gtk_frame_new
       GtkFrame
       ((string label (null-ok))))
\end{verbatim}
and
\begin{verbatim}
    (define-func gtk_label_get
       none
       ((GtkLabel label)
        (string str (output))))
\end{verbatim}

The corresponding generated code is
\begin{itemize}
\item \filename{.c}
\begin{verbatim}
    /* ML type: string option -> gtkobj */
    value mgtk_frame_new(value label) { /* ML */
      return Val_GtkObj(gtk_frame_new(StringOption_nullok(label)));
    }

    /* ML type: unit -> gtkobj */
    value mgtk_frame_new_short() { /* ML */
      return Val_GtkObj(gtk_frame_new(NULL));
    }
\end{verbatim}
and
\begin{verbatim}
    /* ML type: gtkobj -> string */
    value mgtk_label_get(value label) { /* ML */
      char* resstr;
      gtk_label_get(GtkObj_val(label), &amp;resstr);
      return copy_string(resstr);
    }
\end{verbatim}

\item \filename{.sig}
\begin{verbatim}
    val frame_new: string option -> base GtkFrame
    val frame_new': unit -> base GtkFrame
\end{verbatim}
and
\begin{verbatim}
    val label_get: 'a GtkLabel -> string
\end{verbatim}

\item \filename{.sml}
\begin{verbatim}
    val frame_new_: string option -> gtkobj
        = app1(symb"mgtk_frame_new")
    val frame_new: string option -> base GtkFrame
        = fn label => OBJ(frame_new_ label)

    val frame_new'_: unit -> gtkobj
        = app1(symb"mgtk_frame_new_short")
    val frame_new': unit -> base GtkFrame
        = fn label => OBJ(frame_new'_ label)
\end{verbatim}
and
\begin{verbatim}
    val label_get_: gtkobj -> string
        = app1(symb"mgtk_label_get")
    val label_get: 'a GtkLabel -> string
        = fn OBJ label => label_get_ label
\end{verbatim}

\end{itemize}

\subsection{Enumerations/flags}

The \filename{.defs} file contains
\begin{verbatim}
    (define-enum GtkArrowType
       (up GTK_ARROW_UP)
       (down GTK_ARROW_DOWN)
       (left GTK_ARROW_LEFT)
       (right GTK_ARROW_RIGHT))
\end{verbatim}

We need access to the correct integer values of the enumeration
constants. This is provided by \syntax{mgtk_get_arrow_type} (a C
function) that constructs a tuple of all constants, which is bound to
SML values in \filename{Gtk.sml}.

The corresponding generated code is
\begin{itemize}
\item \filename{.c}
\begin{verbatim}
    /* ML type: unit -> int * int * int * int */
    value mgtk_get_arrow_type (value dummy) { /* ML */
      value res = alloc_tuple(4);
      Field(res,0) = Val_int(GTK_ARROW_UP);
      Field(res,1) = Val_int(GTK_ARROW_DOWN);
      Field(res,2) = Val_int(GTK_ARROW_LEFT);
      Field(res,3) = Val_int(GTK_ARROW_RIGHT);
      return res;
    }
\end{verbatim}

\item \filename{.sig}
\begin{verbatim}
    type arrow_type
    val ARROW_UP: arrow_type
    val ARROW_DOWN: arrow_type
    val ARROW_LEFT: arrow_type
    val ARROW_RIGHT: arrow_type
\end{verbatim}

\item \filename{.sml}
\begin{verbatim}
    type arrow_type = int
    val get_arrow_type_: unit -> int * int * int * int
        = app1(symb"mgtk_get_arrow_type")
    val (ARROW_UP,ARROW_DOWN,ARROW_LEFT,ARROW_RIGHT)
        = get_arrow_type_ ()
\end{verbatim}

\end{itemize}

\subsection{Boxed types}

The \filename{.defs} file contains
\begin{verbatim}
    (define-boxed GdkFont
       gdk_font_ref
       gdk_font_unref)
\end{verbatim}

The corresponding generated code is
\begin{itemize}
\item \filename{.c} We need code to convert from the type of ML values
  to C values (\syntax{GdkFont_val}, below) and the other way
  (\syntax{Val_GdkFont}, below). Being a boxed type, we have to
  remember to increase/decrease reference counts with the supplied
  functions.
\begin{verbatim}
    #define GdkFont_val(x) ((void*) Field(x, 1))

    static void ml_finalize_gdk_font (value val) {
      gdk_font_unref (GdkFont_val(val)); 
    }

    value Val_GdkFont (void* obj) {
      value res;
      gdk_font_ref(obj);
      res = alloc_final (2, ml_finalize_gdk_font, 0, 1);
      GdkFont_val(res) = (value) obj ;
      return res;
    }
\end{verbatim}

\item \filename{.sig}
\begin{verbatim}
    type gdk_font
\end{verbatim}

\item \filename{.c}
\begin{verbatim}
    prim_type gdk_font
\end{verbatim}

\end{itemize}

\subsection{Signals}

(Remember that this is non-standard, refer to Section 2.)

The \filename{.defs} file contains
\begin{verbatim}
    (define-signal GtkButton "clicked")
\end{verbatim}


The corresponding generated code is
\begin{itemize}
\item \filename{.c} (nothing---everything is supported by the header
  files; refer to \filename{header.c} and
  \filename{header.sml} for details).

\item \filename{.sig}
\begin{verbatim}
    val connect_clicked: 'a GtkButton -> (unit -> unit) -> unit
\end{verbatim}

\item \filename{.sig}
\begin{verbatim}
    val connect_clicked: 'a GtkButton -> (unit -> unit) -> unit
        = fn wid => fn cb => unit_connect wid "clicked" cb
\end{verbatim}

\end{itemize}

\end{document}

%%% Local Variables: 
%%% mode: latex
%%% TeX-master: t
%%% End: 
